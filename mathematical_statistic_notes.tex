\documentclass[%
	11pt,
	a4paper,
	utf8,
	%twocolumn
		]{article}	

\usepackage{style_packages/podvoyskiy_article_extended}


\begin{document}
\title{Классические и продвинутые темы теории вероятностей и математической статистики}

\author{\itshape Подвойский А.О.}

\date{}
\maketitle

\thispagestyle{fancy}

Здесь приводятся заметки по некоторым вопросам, касающимся машинного обучения, анализа данных, программирования на языках \texttt{Python}, \texttt{R} и прочим сопряженным вопросам так или иначе, затрагивающим работу с данными.


\shorttableofcontents{Краткое содержание}{1}


\tableofcontents

\section{Доверительные интервалы}

\emph{Доверительный интервал} -- интервал, покрывающий неизвестный скалярный параметр $ \theta $ с заданной доверительной вероятностью $ (1 - \alpha) $ (где $ \alpha $ -- уровень значимости).

Границы доверительного интервала являются случайными величинами, поэтому правильнее говорить, что доверительный интервал покрывает неизвестный параметр $ \theta $.



% Источники в "Газовой промышленности" нумеруются по мере упоминания 
\begin{thebibliography}{99}\addcontentsline{toc}{section}{Список литературы}
	\bibitem{lutz:learningpython-2011}{\emph{Лутц М.} Изучаем Python, 4-е издание. -- Пер. с англ. -- СПб.: Символ-Плюс, 2011. -- 1280~с. }
\end{thebibliography}

\end{document}
